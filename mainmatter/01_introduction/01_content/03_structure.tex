\section{Structure of the Thesis}
\label{sec:structure}

Our thesis is divided into ten chapters, whereby this introduction represents the first chapter. It conveys a brief overview of the area of benchmarking REST-based microservices and motivates the pattern-based benchmarking approach.

Our work is based on theoretical concepts and technologies. The second chapter introduces all these concepts and conveys all the knowledge required to understand our work.

Chapters three to seven are the main part of our work and describe the experiment and the results. Chapter three gives a general overview of the experiment, and the subsequent chapters convey a detailed view of the single steps of the experiment and the results.

The last three chapters serve to reflect our own doing. In chapter \ref{cha:discussion} we analyze methods used and assumptions made, we discuss limitations of openISBT and our prototypes, and finally introduce open questions and suggestions for future research. In chapter \ref{cha:related_work} we compare our work to other research and discuss the advantages and disadvantages of the different approaches. In chapter \ref{cha:conclusion}, we finally conclude the experiment and its results.
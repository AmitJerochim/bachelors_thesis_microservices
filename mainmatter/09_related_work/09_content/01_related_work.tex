In this work, we evaluate the applicability of openISBT, a proof-of-concept prototype of the pattern-based benchmarking approach, and extend its applicability. This chapter compares the pattern-based benchmarking approach to prior results in this field. IT-benchmarking is a broad area and covers different fields such as hardware benchmarking or software benchmarking, and each of these fields is broad itself. Both fields can easily overlap. For example, a common microservice might include a data storage system and run in a container, which runs on a cluster. Therefore, a clear separation of all kinds of SUT is difficult, so we consider more kinds of systems than just microservices in this section. Also, there are benchmarks for measuring several non-functional properties, and to narrow down our research field by the SUT is not necessarily the best approach. For example, a security benchmark for DBMS and a security benchmark for REST APIs might have more in common than a security benchmark and a performance benchmark for REST APIs.

Therefore, this section discusses several fields of research.
There are different virtual machines and containers benchmarking approaches \cite{RelWor:VM_Borhani, RelWor:VM_Kejiang, RelWor:VM_Varghese, RelWor:VM_Zaitsev, RelWor:VM_Carpio} that introduce powerful tools for measuring different virtualization scenarios. There are several approaches for data storage system benchmarking \cite{RelWor:DB_NDBench, YCSB+T, BenchFoundary, RelWor:DB_COSBench, RelWork:OLTP, RelWork:CloudDBFramework, YCSB++}, web-application benchmarking  \cite{bermbach_benchmarking_web_api, bermbach_benchmarking_web_api_rev}, and also application-driven benchmarking approaches \cite{RelWor:VM_containerized_webapps, deathStarBench, relwork_µ-suite}. However,  openISBT appears to be the only approach that provides a generic benchmark for microservices.


Similar to our approach, Zheng et al. also use interaction patterns\cite{RelWor:DB_COSBench}. Their approach is only designed for Cloud Object Storage. Therefore, it relies on CDMI, an interface based on RESTful principles but is less heterogeneous than REST.

Steven Bucaille et al. use OpenAPI to measure non-functional properties of REST APIs \cite{openAPI_Monitoring_framework}. In particular, they measure the performance and availability of services in different geographical locations by means of a master/slave architecture. However, their approach focuses on the monitoring of REST APIs and not on benchmarking. Also, their approach is not pattern-based but handles each operation the same way. Stefan Karlsson et al. use OpenAPI to dynamically generate tests for functional properties \cite{OpenAPI_testing_framework} and run them against the implementation from a client's perspective. Their research does not belong to the area of IT-benchmarking or even the measurement of non-functional properties of systems. However, they discuss the difficulty of resolving parameters under the term "stateful sequences" and implicitly use interaction patterns. 

% When we started our research on this topic OpenISBT implemented matching units for services that map to CRUD and also implemented a workload generator to measure the performance of a system. We extended openISBT to also map services to abstract operations for authentication, authorization and request permissioning. Focusing on security issues our extended matching tool could be used in combination with attack executors such as  \cite{relwork_Mendes_security} instead of a workload generator. 

Section \ref{sec:bg_microservices} mentions RabbitMQ and ZeroMQ as alternatives to REST. Currently, openISBT can not benchmark microservices using these mechanisms, which is related to the fact that the OpenAPI specification is not a proper IDL for them. However, Xian Jun Hong et al. compare REST-based and RabbitMQ-based microservices and show that for a high workload, asynchronous mechanisms have an advantage over REST APIs \cite{relWork:comparing_RabbitMQ_and_REST}.
OpenAPI evolved in the last decade to a de facto standard among the RESTish IDLs and is discussed in several academic papers. At the same time, asynchronous APIs such as RabbitMQ and ZeroMQ had neither an IDL nor all products built around it (e.g., stub generator). 
AsyncAPI\footnote{\url{https://www.asyncapi.com/}} is a new machine-readable interface description language for asynchronous APIs that reached academia in the year 2020. To the best of our knowledge, there is only one academic paper where Abel G\'omez et al. evaluate the usage of AsyncAPI for application development purposes \cite{relWork:AyncAPI}. Nonetheless, AsyncAPI allows usage of YAML and JSON, as OpenAPI does. Both languages' syntax is very similar, which reduces the effort to extend openISBT to benchmark event-driven and asynchronous microservices.
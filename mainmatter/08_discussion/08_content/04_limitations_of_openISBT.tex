\section{Limitations of the OpenISBT Implementation}
\label{sec:discussion_limitations_implementation}

The openISBT implementation has issues that the authors' approach does not affect, so we discuss the prototype's limitations independently of the approach's limitations. Section \ref{sec:analysis1:larger_context_of_applicability} already discusses the implications of nested resources in OpenAPI documents and poor exception handling. In our work, we notice a third limitation regarding the openISBT implementation. The approach assumes that each service operation is matched to exactly one abstract operation, but we notice that a service operation can be matched to multiple abstract operations. The authors fix this issue by prioritizing abstract operations. For instance, SCAN is more specific than READ and therefore has a higher priority. This approach works well for many service operations but might also lead to false-positive errors. Our modifications to the matching units in section \ref{sec:extending:improving_existing_matching_units} solve this issue for SCAN and READ without prioritization by checking mutually exclusive properties. We cannot find mutually exclusive properties for the abstract operations VALIDATE, and READ, so we prioritize VALIDATE over READ. However, prioritizing VALIDATE might be a source of false-positive errors because openISBT might match service operations to VALIDATE, although they should match them to READ.
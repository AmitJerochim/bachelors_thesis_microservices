\subsection{The CONTROL Abstract Operation}

This abstract operation describes service operations where a client requests other service operation's metadata. The use cases for such an operation vary, so we present different examples.
The first example assumes that a client considers re-downloading a large resource. Using a CONTROL operation, it could check if the resource has been modified since the last download. In scenarios where bandwidth is limited, the client might also check the resource's content-length before requesting it.

In some scenarios, clients require information about accepted requests. To receive this information, the client would perform a preflight request to the server. For most preflight requests, developers do not need to implement the service operations explicitly. For example, if a client uses a preflight request to ask the server if it would allow an HTTP DELETE request. For some preflight requests, explicitly defining and implementing the service operation is required. For example, if a client sends a cross-origin request including cookies or non-standard headers to domain A from a script served by domain B. 

A client needs to perform an HTTP HEAD request to only ask for the resource's metadata. Browsers use the HTTP OPTIONS method to perform preflight requests. Both HTTP methods are safe and idempotent and allow neither request body nor response body.  


\begin{longtable}{ m{2.6cm} | m{4.4cm} | m{6.5cm} }
\caption{Search syntax for SwaggerHub} 
\label{tab:search_syntax_swaggerhub} \\

\multicolumn{1}{c}{\textbf{Parameter}} & 
\multicolumn{1}{c}{\textbf{Possible Values}} &
\multicolumn{1}{c}{\textbf{Explanation}} \\ 
\toprule
\endfirsthead

\multicolumn{2}{c}%
{\tablename\ \thetable{} -- continued from previous page} \\ 
\multicolumn{1}{c}{\textbf{Parameter}} &
\multicolumn{1}{c}{\textbf{Possible Values}} & \multicolumn{1}{c}{\textbf{Explanation}} \\ 
\toprule
\endhead

\midrule
\multicolumn{2}{r}{{Continued on next page}} \\
\endfoot

\bottomrule
\endlastfoot
\hline
Specification & 
\begin{itemize}
    \item OPENAPI3.0
    \item SWAGGER2.0   
\end{itemize} & 
SwaggerHub hosts OpenAPI 2.0 formatted documents (formerly known as Swagger 2.0) and OpenAPI 3.0 formatted documents. Using this parameter, we can specify the required format.

\\
\hline
Visibility & \begin{itemize}
    \item PUBLIC
    \item PRIVATE  
\end{itemize} & 
The visibility shows if a document is either private or public. Private means that a document is only visible to its owner and contributors. Public documents are visible to anybody, i.e., also to unregistered users.
\\
\hline
State &
\begin{itemize}
    \item PUBLISHED
    \item UNPUBLISHED  
\end{itemize} &
The state shows whether an API is ready for use in production or not. Setting the state to published signals that an API is stable and makes it also read-only.
\\ 
\hline
Type & 
\begin{itemize}
    \item API
    \item DOMAIN  
\end{itemize} &
The type shows whether a document is an API description or a domain. A domain is a library of common components used in multiple API descriptions, e.g., a user object's data model.  
\\
\hline
Page & 
Any page number starting with one & 
SwaggerHub uses pagination to limit requested results. The client only gets the first ten items per request. Using this query string parameter, we can request additional items. \\

\end{longtable}